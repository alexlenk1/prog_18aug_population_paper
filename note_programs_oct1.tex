\documentclass[11pt,fleqn]{article}

\usepackage{amsfonts,ulem,url}
\usepackage{amssymb}
\usepackage{amsmath}
\usepackage{amstext,verbatim,graphicx,ifthen,multirow,amsthm,color,setspace}
%\usepackage{subfigure}

\onehalfspacing

\topmargin=-0.7in
\textheight=9.0in
\oddsidemargin=-0.15in
\textwidth=6.5in


\begin{document}

\centerline{Comments on Simulation Programs}

Here I am following the notation in the matlab code, with parenthetical comments about the corresponding concepts in the paper. 


For each of the designs there are $n$ units in the population. For notational brevity, we omit the variable subscript $n$ everywhere in this note, although it remains implicit that all variables are generated with respect to population $n$. 

In this simulation study, we focus on a single causal variable $U_i$. Each unit $i$ is characterized by three variables, $\xi_i$, $\theta_i$ and $Z_i$.  The non-stochastic pair $(\xi_i, \theta_i)$ summarizes unit $i$'s potential outcome dependence on observed and unobserved attributes. In our setting, the fixed observed attributes are included in $Z_i$, a $KZ+1$ dimensional vector with the first element equal to 1, and the remaining $KZ$ elements drawn from a normal distribution with mean zero and unit variance, all independent.
The population is generated first, with:
\[ Z_{i1}=1,\hskip1cm Z_{ik}\sim{\cal N}(0,1).\]
\[ \xi_i\sim {\cal N}(Z_i^\top \gamma,\sigma_\xi^2)\]
\[ \theta_i\sim {\cal N}(Z_i^\top \psi,\sigma_\theta^2).\]

In this particular simulation study, we assume all throughout $\sigma_\xi = 1, \gamma = 0$ such that $\xi_i\sim {\cal N}(0,1)$. Having generated our population of interest, the potential outcome function for unit $i$ with $U_i = u$ is given as in (3.6) by:
\[ Y_i(u)= \theta_i u_i + \xi_i. \] 
The assignment mechanism is
\[ U_i\sim {\cal N}(Z_i^\top\lambda,1).\]
We work with the potential outcome function in terms of the causal variable $X_i$ which is free of the correlation from $Z_i$. Since for our simulations, we have $\mathbb{E}[U_i] = 0$, by (3.1), this implies that $\lambda = 0$, such that it holds $X_i = U_i$, and we can directly re-write the potential outcome function as \footnote{The way this is implemented in the code is by the direct assumption that $U_i$ and $Z_i$ are uncorrelated, i.e., by directly imposing $\lambda = 0$ rather than by assuming $\mathbb{E}[U_i] = 0$.}:
\[ Y_i(u)= x_i' \theta_i + \xi_i. \] 


Once we have this population, we can calculate the various estimands and the various variances.
Because $X_i$ and $Z_i$ are uncorrelated, for the limit of the estimands we have
\[ \lim_{n\rightarrow\infty} \theta_n=\lim_{n\rightarrow\infty}
\frac{1}{n}\sum_{i=1}^n \phi_i=\mathbb{E}[Z]^\top \psi=\psi_1=0\]
in all the designs.
\[\lim_{n\rightarrow\infty} \gamma_n=\mathbb{E}[Z_iZ^\top_i]^{-1}\mathbb{E}[Z_iY_i]=\mathbb{E}[Z_iY_i].\]
For the first element of the limit of $\gamma_n$, corresponding to the intercept, we have
\[ \lim_{n\rightarrow\infty}\gamma_{1n}=\mathbb{E}[Y_i]=
\mathbb{E}[\alpha_i]+\mathbb{E}[\phi_iX_i]
=0+\mathbb{E}[Z_i^\top \psi_iX_i]+\mathbb{E}[(\phi_i-Z_i^\top \psi)X_i]=0\]
For the other elements
\[ \lim_{n\rightarrow\infty}\gamma_{kn}=\mathbb{E}[Z_{ik}Y_i]=
\mathbb{E}[Z_{ik}\alpha_i]+\mathbb{E}[Z_{ik}\phi_iX_i]\]
\[
=\mathbb{E}[Z_{ik}Z_i^\top\gamma]+\mathbb{E}[Z_{ik}(\alpha_i-Z_i^\top\gamma)]+\mathbb{E}[Z_{ik}Z_i^\top \psi_iX_i]+\mathbb{E}[Z_{ik}(\phi_i-Z_i^\top \psi)X_i]=0\]
$===========================================$

Next, we simulate samples. This involves samping units from the population, and assigning them causes $X_i$ to the units sampled.

\vskip0.5cm

The list of programs used in the simulations is
\begin{enumerate}
  \item {\tt main\_18aug8.m} This is the main program that runs the simulations.
  \item {\tt gen\_potential.m} This program generates the population. It takes as input the population size $n$, the $KZ+1$ vector $\psi$, the $KZ+1$ vector $\gamma$, the $KZ+1$ vector $\xi$, and the scalars $\sigma_\varepsilon$, and $\sigma_\eta$, and the integer $KZ$, and puts out the $KZ+1$ vector $Z_i$, and the scalars $\alpha_i$, and $\phi_i$, for $i=1,\ldots,N$.
 \begin{enumerate}
\item $KZ$ is the number of characteristics, beyond the intercept. In the first design $KZ=1$. 
\item $\xi$ is not used
\item In the first design, $\gamma=(0,0)^\top$.
\item In the first design, $\psi=(0,2)^\top$.
\item $\alpha_i\sim{\cal N}(Z_i^\top\gamma,\sigma_\eta^2)$
\item $\phi_i\sim{\cal N}(Z_i^\top\psi,\sigma_\varepsilon^2)$\end{enumerate}
\item {\tt gen\_sample.m} This program takes the population, characterized by $(Z_i$, $\alpha_i$, $\phi_i$, and samples a fraction $\rho$ from this population, and assigns a value of $X_i$ to them. 
The input $\xi$ ???
The output is
\begin{enumerate}
\item $YR$
\item $XR$
\item $ZR$
\item $W$
\item $UR$
\item $R$
\item $\Omega$
\item $\theta_{\rm desc}$
\item $\theta_{\rm causal,sample}$
\end{enumerate}  

\end{enumerate}

\end{document}


